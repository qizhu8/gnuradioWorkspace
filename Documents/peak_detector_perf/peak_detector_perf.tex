\documentclass[a4paper]{article}
\usepackage{algorithmic}
\usepackage{array}
\usepackage{url,amsfonts,amsmath,amssymb,amsthm,color,enumerate,tikz,hyperref}
\usepackage{algorithm,bbm}
\usepackage{mathtools}
% \usepackage{indentfirst}
\usepackage{stackrel}
\usepackage{listings}
\usepackage{subfigure}
\usepackage{subfloat}
\usepackage[justification=centering]{caption}
\usepackage{booktabs} % for three-line table
% \usepackage{subfig}
\usepackage{graphicx}
\hyphenation{op-tical net-works semi-conduc-tor}
% \usepackage[english]{babel}
% \usepackage[utf8]{inputenc}
% \usepackage{amsmath}
% \usepackage{graphicx}
% \usepackage[colorinlistoftodos]{todonotes}

\title{Mathematical Model of Input for Peak-Detector}

\author{Yu}

\date{\today}



\begin{document}

% \author{Yu Wang% <-this % stops a space
% % \thanks{}% <-this % stops a space
% }

% The paper headers
% \markboth{}%
% {}
\maketitle
% \IEEEpeerreviewmaketitle

% def some function
\DeclarePairedDelimiter\ceil{\lceil}{\rceil}
\DeclarePairedDelimiter\floor{\lfloor}{\rfloor}

\section{Introduction} % (fold)
\label{sec:introduction}
For the ``freq\_timing\_est'' block, frequency and timing information are supposed to be estimated. As discussed in the ``freq\_timing\_tradeoff'', we can use brutal search to get the correct value. But, before we make our decision by choosing the maximum peak, we still need to have a rigorous mathematic model for the peak.

This article will discuss this problem step to step.
% section introduction (end)


\section{Mathematical Model} % (fold)
\label{sec:mathematical_model}
Because the signal we received are discrete values, so we only consider about discrete signals, rather than continuous ones.

\begin{table}[ht]
\centering
\begin{tabular}{l l}
 \toprule
 Notation 	& 	Description\\
 \midrule
 $s_k$		& 	received sample at time $k$\\
 $t_k$		& 	training sample at time $k$\\
 $d_k$		& 	data sample at time $k$\\
 $n_k$		& 	noise at time $k$\\
 $P^t_k$ & 	the $k^{th}$ chip of spread code for one frame of training\\
 $P^d_k$	& 	the $k^{th}$ chip of spread code for one frame of data\\
 $D^t_l$ & 	the $l^{th}$ symbol for training\\
 $D^d_l$ & 	the $l^{th}$ symbol for data\\
 $N_f$ 		& 	number of chips per frame\\
 $N_c$ 		& 	number of chips per symbol\\
 $N_s$		& 	number of symbols per frame\\
 $N_{s'}$ 	& 	number of symbols per shortened training sequence\\
 $\sigma^2$ & 	variance of Gaussian noise\\
 $k_0$ 		& 	number of samples of delay\\
 $f_0$ 		& 	frequency offset\\
 $T_c$ 		& 	duration between two chips\\
 \bottomrule
\end{tabular}
\caption{Notation Table}
\label{Table:Notation Table}
\end{table}

First, the structure of training and data frame are like:
The structure of training is like Fig. \ref{fig:Training Frame Structure} and Fig. \ref{fig:Data Frame Structure}
\begin{figure}[ht]
	\centering
	\includegraphics[width = 3.1in]{figure/training_frame.png}
	\caption{Training Frame Structure}
	\label{fig:Training Frame Structure}
	\includegraphics[width = 3.1 in]{figure/data_frame.png}
	\caption{Data Frame Structure}
	\label{fig:Data Frame Structure}
\end{figure}

\subsection{Description of Variables} % (fold)
\label{sub:description_of_variables}
We first express each variables by mathematical languages.

According to the description in table \ref{Table:Notation Table}, and also in order to simplify the following notation, the definition of $P^x_y$ is defined as 
\begin{equation}
\begin{split}
	&P^x_k = 
	\begin{cases}
		P^x_k 	& k = 0, \cdots, N_c-1\\
		0 			& else
	\end{cases}\\
	&x = ``t'' or ``d''
\end{split}
\end{equation}

For instance, the figure for $P^t_k$ is like Fig. \ref{fig:Sketch Map of P_tk}
\begin{figure}
	\centering
	\includegraphics[width = 3.1 in]{figure/p_t_k.png}
	\caption{Sketch Map of $P^t_k$}
	\label{fig:Sketch Map of P_tk}
\end{figure}

So, training chip at time $k$ can be described as:
\begin{align}
		t_k 
	&= \sum_{k = 0}^{N_c} D^t_0 P^t_{k}\\
	&= D^t_0 P^t_k \label{eq:t_k only consider the cyclic property of symbol level}
\end{align}

If we consider the frame-level periodicity, $t_k$ would be
\begin{align}
	t_k = D^t_{\floor{k / N_c} \% N_s} P^t_{k \% N_c}
\end{align}

Based on nearly the same idea, $d_k$ can be written as:
\begin{align}
	d_k = D^d_{\floor{k / N_c}} P^d_{k \% N_c}
\end{align}

Noise is simple. We simply choose IID Gaussian noise. So
\begin{align}
	n_k \sim \mathcal{N}(0, \sigma^2)
\end{align}
% subsection description_of_variables (end)

\subsection{Description of Output of Matched Filter} % (fold)
\label{sub:description_of_output_of_matched_filter}
Consider the time delay and frequency offset, the received signal would be 
\begin{align}
	r_k = (t_{k - k_0} + d_{k - k_0}) e^{j 2\pi f_0 k T_c} + n_k
\end{align}

We first deal with the contribution of noise.
Define the Pulse Response of Matched Filter to be $h_{N_f-k}$.
\begin{align}
	h_k = 
	\begin{cases}
	P^t_{k} = D^t_{\floor{k/N_c}}P^t_{k\%N_c} & k = 0, \cdots, N_f\\
	0 	& else	
	\end{cases}
\end{align}
Let $g^n_{k_0}= \sum_{k' = 0}^{N_f-1} n_{k'} \cdot h_{k'}$

\begin{align}
	\mathbb{E} \{ g^n_{k_0}\}
	&= \mathbb{E} \{\sum_{k' = 0}^{N_f-1} n_{k'} \cdot h_{k'}\} \\
	&= \sum_{k' = 0}^{N_f-1} \mathbb{E} \{ n_{k'} \cdot h_{k'} \} \\
	&= \sum_{k' = 0}^{N_f-1} \mathbb{E} \{ n_{k'}\} h_{k'} \\
	&= \sum_{k' = 0}^{N_f-1} 0 \cdot h_{k'} \\
	&= 0\\
	\mathbb{E} \{ (g^n_{k_0})^2\}
	&= \mathbb{E} \{\sum_{k' = 0}^{N_f-1} (n_{k'} \cdot h_{k'}\} \\
	&= \mathbb{E} \{\sum_{k' = 0}^{N_f-1} n_{k'}^2 \cdot \underbrace{(h_{k'})^2}_{\equiv1}\} \\
	&= \sum_{k' = 0}^{N_f-1} \mathbb{E} \{n_{k'}^2 \}\\
	&= N_f \sigma^2
\end{align}

So $g^n_{k_0}\sim \mathcal{N} (0, N_f \sigma^2)$

The next step is to analyze the output of Matched Filter. 
Define the patten for Matched Filter for training frame to be $P^{t*}_{-k}$.
The output of the Matched Filter at sampled point $k=N_f$ is:
\begin{align}
	g_{k_0} 
	&= r_k \ast h_{N_f - k} \bigg|_{k = N_f-1}\\
	&= (t_{k - k_0} + d_{k - k_0} + n_k) \ast h_{N_f -1- k} \bigg|_{k = N_f-1}\\
	&= \underbrace{\sum_{k' = 0}^{N_f-1} t_{k'-k_0}\cdot h_{k'}}_{g^t_{k_0}} + \underbrace{\sum_{k' = 0}^{N_f-1} d_{k'-k_0}\cdot h_{k'}}_{g^d_{k_0}} + \underbrace{\sum_{k' = 0}^{N_f-1} n_{k'}\cdot h_{k'}}_{g^n_{k_0}}\\
	&= g^t_{k_0} + g^d_{k_0} + g^n_{k_0}
	% g_{k_0}
	% &= r_k \ast h_{N_f-k} \bigg|_{k = N_f}\\
	% &= \sum_{k' = 0}^{N_f-1} r_{k'} h_{N_f-(k - k')}\bigg|_{k = N_f}\\
	% &= \sum_{k' = k-N_f+1}^{k} r_{k'} \cdot h_{k' + (N_f-k)}\bigg|_{k = N_f}\\
	% &= \sum_{k' = 0}^{N_f-1} (t_{k'-k_0-N_f+1} + d_{k'-k_0-N_f+1} + n_{k'-k_0-N_f+1}) \cdot h_{k'}\\
	% &= \sum_{k' = 0}^{N_f-1} (t_{k'-k_0-N_f+1} + d_{k'-k_0-N_f+1}) \cdot h_{k'} + \sum_{k' = 0}^{N_f-1} n_{k'-k_0-N_f+1} \cdot h_{k'}\\
	% &= \sum_{k' = 0}^{N_f-1} (t_{k'-k_0-N_f+1} + d_{k'-k_0-N_f+1}) \cdot h_{k'} + \sum_{k' = 0}^{N_f-1} n_{k'} \cdot h_{k'}\\
	% &= \sum_{k' = 0}^{N_f-1}  (t_{k'-k_0-N_f+1} + d_{k'-k_0-N_f+1}) \cdot h_{k'} + g^n_{k}\\
	% &= \underbrace{\sum_{k' = 0}^{N_f-1} t_{k'-k_0-N_f+1} \cdot h_{k'}}_{g^t_{k_0}} + \underbrace{\sum_{k' = 0}^{N_f-1} d_{k'-k_0-N_f+1} \cdot h_{k'}}_{g^d_{k_0}} + g^n_{k_0}\\
	% &= g^t_{k_0}+ g^d_{k_0}+ g^n_{k_0}
\end{align}

Now, we will discuss all the possible situations for timing. 
Let's formate the time offset $k_0$ in terms of $N_s$ and $N_c$.
\begin{align}
	k_0 = \alpha \cdot N_c + \beta ~~~ \alpha = \mathbb{Z}~and~\beta = 0, \cdots, N_c - 1
\end{align}



\subsubsection{Perfectly Aligned} % (fold)
\label{ssub:perfectly_aligned}
Perfectly Aligned means the sample at time $k$ is also the first sample of patten for Matched Filter, as shown in fig. \ref{fig:Perfectly Aligned}.
\begin{figure}[ht]
	\centering
	\includegraphics[width=3.1in]{figure/perfectly_aligned.png}
	\caption{Perfectly Aligned}
	\label{fig:Perfectly Aligned}
\end{figure}
For this situation, $k_0$ satisfies 
\begin{align}
	\begin{cases}
		\alpha = z \cdot N_s ~ z \in \mathbb{Z}\\
		\beta = 0
	\end{cases}
	\Rightarrow
	k_0 = z \cdot N_s \cdot N_c
\end{align}

Then we can analyze $g^t_{k_0}$ and $g^d_{k_0}$ separately
\begin{align}
	g^t_{k_0}
	&= \sum_{k' = 0}^{N_f-1} t_{k'-k_0} \cdot h_{k'}\\
	&= \sum_{k' = 0}^{N_f-1} D^t_{\floor{(k'-k_0) / N_c}\%N_s} P^t_{k' \% N_c} \cdot h_{k'}\\
	&= \sum_{k' = 0}^{N_f-1} D^t_{\floor{(k'-z \cdot N_s \cdot N_c) /N_c}\% N_s} P^t_{k' \% N_c} \cdot h_{k'}\\
	&= \sum_{k' = 0}^{N_f-1} D^t_{\floor{k'/N_c}\%N_s} P^t_{k' \% N_c} \cdot h_{k'}\\
\end{align}
\begin{align}
	\because 
	h_{k'} &= D^t_{\floor{k'/N_c}} P^t_{k' \% N_c}\\
	\therefore
	g^t_{k_0}
	&= \sum_{k' = 0}^{N_f-1} D^t_{\floor{k'/N_c}\%N_s} P^t_{k' \% N_c} \cdot D^t_{\floor{k'/N_c}} P^t_{k \% N_c}\\
	&= \sum_{k' = 0}^{N_f-1} \underbrace{(D^t_{\floor{k'/N_c}\%N_s})^2}_{\equiv 1} \underbrace{(P^t_{k' \% N_c})^2}_{\equiv1}\\
	&= N_f
\end{align}

To analyzie $g^s_k$, we will first define a function to describe the convolution
\begin{align}
	C^n_s(f_1, f_2, k_0) = \sum_{k = s}^{s+n} f_1(k)\cdot f_2(k - k_0) \label{eq:New Defined Convolution}
\end{align}

The diagram of the above formula is like Fig.\ref{fig:New Defined Convolution}
\begin{figure}[ht]
	\centering
	\includegraphics[width=3.1in]{figure/conv_figure.png}
	\caption{New Defined Convolution}
	\label{fig:New Defined Convolution}
\end{figure}
\begin{align}
	g^d_{k_0}
	&= \sum_{k' = 0}^{N_f-1} d_{k' - k_0} \cdot h_{k'}\\
	&= \sum_{k' = 0}^{N_f-1} D^d_{\floor{(k' - k_0)/N_c}}P^d_{(k' - k_0)\%N_c} \cdot h_{k'}\\
	&= \sum_{k' = 0}^{N_f-1} D^d_{\floor{(k' - z \cdot N_s \cdot N_c)/N_c}} P^d_{(k' - z \cdot N_s \cdot N_c)\%N_c}\cdot h_{k'}\\
	&= \sum_{k' = 0}^{N_f-1} D^d_{\floor{k'/N_c} - z \cdot N_s} P^d_{k'\%N_c} \cdot D^t_{\floor{k'/N_c}} P^t_{k' \% N_c}\\
\end{align}
For data symbols, we could only assume that 
\begin{align}
	D^d_k = 
	\begin{cases}
	+1 & p = 0.5\\
	-1 & p = 0.5	
	\end{cases}
\end{align}
Because the $D^d_k$ here is a random variable, we could only exam the mean and variance for it, so do the $g^d_{k_0}$
\begin{align}
	&\mathbb{E} \{D^d_k\} = 1 \times 0.5 + (-1) \times 0.5 = 0\\
	&\mathbb{E} \{D^d_k)^2\} = 1 \times 0.5 + (-1)^2 \times 0.5 = 1
\end{align}
and they are IID.

For $g^d_{k_0}$,

\begin{align}
	% mean
	\mathbb{E} \{g^d_{k_0}\} 
	&= \mathbb{E} \{\sum_{k' = 0}^{N_f-1} D^d_{\floor{k'/N_c} - z \cdot N_s} P^d_{k'\%N_c} \cdot D^t_{\floor{k'/N_c}} P^t_{k' \% N_c}\}\\
	&= \sum_{k' = 0}^{N_f-1} \mathbb{E}\{ D^d_{\floor{k'/N_c} - z \cdot N_s} P^d_{k'\%N_c} \cdot D^t_{\floor{k'/N_c}} P^t_{k' \% N_c}\}\\
	&= \sum_{k' = 0}^{N_f-1} P^d_{k'\%N_c} P^t_{k' \% N_c} \mathbb{E} \{ D^d_{\floor{k'/N_c} - z \cdot N_s} D^t_{\floor{k'/N_c}}\}\\
	&= C_0^{N_c-1}(P^d_{k}, P^t_{k}, 0) \sum_{k'' = \floor{k'/N_c}}^{\floor{k'/N_c} + N_s - 1} D^t_{k''}\mathbb{E} \{ D^d_{k'' - z} \}\\
	&= 0
\end{align}
\begin{align}
	% variance
	\mathbb{E} \{(g^d_{k_0})^2\}
	&= \mathbb{E} \{\sum_{k' = 0}^{N_f-1} D^d_{\floor{k'/N_c} - z \cdot N_s} P^d_{k'\%N_c} \cdot D^t_{\floor{k'/N_c}\%N_s} P^t_{k \% N_c}\}^2\\
	&= \mathbb{E} \{\sum_{k'' = \floor{k'/N_c}}^{\floor{k'/N_c} + N_s - 1} D^t_{k''} D^d_{k'' - z} \cdot \sum_{m = 0}^{N_c} P^d_{m\%N_c} P^t_{m \% N_c} \}^2\\
	&= \big(\sum_{m = 0}^{N_c} P^d_{m} P^t_{m }\big)^2 \mathbb{E} \{\sum_{k'' = \floor{k'/N_c}}^{\floor{k'/N_c} + N_s - 1} D^t_{k''} D^d_{k'' - z}\}^2\\
	&= (C_0^{N_c-1}(P^d_{k}, P^t_{k}, 0))^2 \mathbb{E}\{\sum_{k'' = \floor{k'/N_c}}^{\floor{k'/N_c} + N_s - 1} D^t_{k''} D^d_{k''}\}^2\\ 
	\begin{split}
		&=(C_0^{N_c-1}(P^d_{k}, P^t_{k}, 0))^2 \bigg(\mathbb{E} \{\sum_{m = \floor{k'/N_c}}^{\floor{k'/N_c} + N_s - 1} (D^t_{m} D^d_{m})^2\}\\
		&+ \mathbb{E} \{ \sum_{m = \floor{k'/N_c}}^{\floor{k'/N_c} + N_s - 1} \sum_{n \neq m} D^t_{m} D^d_{m} D^t_{n} D^d_{n}\} \bigg)
	\end{split}\\
	\begin{split}
		&= (C_0^{N_c-1}(P^d_{k}, P^t_{k}, 0))^2 \bigg( \sum_{m = \floor{k'/N_c}}^{\floor{k'/N_c} + N_s - 1} \mathbb{E} \{ D^d_{m}\}^2\\
		&+ \sum_{m = \floor{k'/N_c}}^{\floor{k'/N_c} + N_s - 1} \sum_{n \neq m} D^t_{m} D^t_{n} \mathbb{E} \{D^d_{m}\} \mathbb{E} \{D^d_{n}\} \bigg)
	\end{split}\\
	&= (C_0^{N_c-1}(P^d_{k}, P^t_{k}, 0))^2 \bigg( \sum_{m = \floor{k'/N_c}}^{\floor{k'/N_c} + N_s - 1} \cdot 1 \bigg)\\
	&= (C_0^{N_c-1}(P^d_{k}, P^t_{k}, 0))^2 \cdot N_s
\end{align}
% subsubsection perfectly_aligned (end)

\subsubsection{Chip-level Aligned} % (fold)
\label{ssub:chip_level_aligned}
Chip-level Aligned means the start chip of received symbol is aligned with patten's symbol's first chip. But, the begining symbol of a frame for received data may not be aligned with patten's first symbol. Here is the diagram
\begin{figure}[ht]
	\centering
	\includegraphics[width = 3.1in]{figure/chip_level_aligned.png}
	\caption{Chip-level Aligned}
	\label{fig:Chip-level aligned}
\end{figure}

At this time
\begin{align}
	\begin{cases}
		\alpha = z &z \in \mathbb{Z}\\
		\beta = 0 &
	\end{cases}
	\Rightarrow k_0 = z \cdot N_c
\end{align}

For $g^t_{k_0}$:
\begin{align}
	g^t_{k_0}
	&= \sum_{k' = 0}^{N_f-1} t_{k'-k_0} \cdot h_{k'}\\
	&= \sum_{k' = 0}^{N_f-1} D^t_{\floor{(k'-k_0) / N_c}\%N_s} P^t_{k \% N_c} \cdot h_{k'}\\
	&= \sum_{k' = 0}^{N_f-1} D^t_{\floor{(k'-z \cdot N_c) / N_c}\%N_s} P^t_{k \% N_c} \cdot D^t_{\floor{k'/N_c}} P^t_{k \% N_c}\\
	&= \sum_{k' = 0}^{N_f-1} D^t_{(\floor{k' / N_c} - z)\%N_s} D^t_{\floor{k'/N_c}} \underbrace{P^t_{k \% N_c} P^t_{k \% N_c}}_{\equiv 1}\\
	&= \sum_{k' = 0}^{N_f-1} D^t_{(\floor{k' / N_c} - z)\%N_s} D^t_{\floor{k'/N_c}}\\
	&= C_z^{N_s-1}(D_k^t, D_k^t, z) + C_{N_s - z}^{N_s-1}(D_k^t, D_k^t, N_s - z)
\end{align}
\begin{figure}[ht]
	\centering
	\includegraphics[width=3.1in]{figure/misaligned_conv.png}
	\caption{Diagram for Misaligned Convolution}
	\label{fig:Diagram for Misaligned Convolution}
\end{figure}

For $g^d_{k_0}$:
\begin{align}
	% mean
	\mathbb{E} \{g^d_{k_0}\} 
	&= \mathbb{E} \{\sum_{k'=0}^{N_f-1} D^d_{\floor{k'/N_c}-z} P^d_{k'\%N_c} \cdot D^t_{\floor{k'/N_c}\%N_s} P^t_{k\%N_c} \}\\
	&= \sum_{k'=0}^{N_f-1} \mathbb{E} \{ D^d_{\floor{k'/N_c}-z} P^d_{k'\%N_c} \cdot D^t_{\floor{k'/N_c}\%N_s} P^t_{k\%N_c} \}\\
	&= \sum_{m = 0}^{N_c} (P^d_{m} P^t_{m}) \sum_{m = \floor{k'/N_c}}^{\floor{k'/N_c}+N_s-1} D^t_{m} \mathbb{E}\{ D^d_{m-z} \}\\
	&= C_0^{N_c}(P_k^d, P_k^t, 0) \cdot \sum_{m = \floor{k'/N_c}}^{\floor{k'/N_c}+N_s-1} D^t_{m} \cdot0\\
	&= 0
\end{align}
\begin{align}
	% variance
	\mathbb{E} \{(g^d_{k_0})^2\}
	&= \mathbb{E} \{\sum_{k'=0}^{N_f-1} D^d_{\floor{k'/N_c}-z} P^d_{k'\%N_c} \cdot D^t_{\floor{k'/N_c}\%N_s} P^t_{k\%N_c} \}^2\\
	&= (C_0^{N_c}(P_k^d, P_k^t, 0))^2 \mathbb{E}\{\sum_{m = \floor{k'/N_c}}^{\floor{k'/N_c}+N_s-1} D_{m-z}^t D_{m}^d \}^2\\
	&= (C_0^{N_c}(P_k^d, P_k^t, 0))^2 \mathbb{E}\{\sum_{m = \floor{k'/N_c}}^{\floor{k'/N_c}+N_s-1} D_{m}^t D_{m}^d \}^2\\
	&= (C_0^{N_c-1}(P^d_{k}, P^t_{k}, 0))^2 \cdot N_s
\end{align}

The mean and variance of $g^d_{k_0}$ is the same as the situation of perfectly aligned.
% subsubsection chip_level_aligned (end)

\subsubsection{Totally Misaligned} % (fold)
\label{ssub:totally_misaligned}
Totally Misaligned means the first chip of received data is not correspond to the first chip of patten. Like the Fig. \ref{fig:Totally Misaligned}
\begin{figure}[ht]
	\centering
	\includegraphics[width = 3.1in]{figure/totally_misaligned.png}
	\caption{Totally Misaligned}
	\label{fig:Totally Misaligned}
\end{figure}

\begin{align}
	k_0 = z, ~ z\in \mathbb{Z}
\end{align}

For $g_{k_0}^t$:
\begin{align}
	g^t_{k_0} 
	&= \sum_{k' = 0}^{N_f-1} t_{k' - k_0} \cdot h_{k'}\\
	&= \sum_{k' = 0}^{N_f-1} D^t_{\floor{(k'-z)/N_c}\%N_s} P^t_{(k'-z)\% N_c} \cdot D^t_{\floor{k'/N_c}} P^t_{k\%N_c}
\end{align}

We will first deal with a smaller problem. If we only focus on patten of length $N_c$
\begin{figure}[ht]
	\centering
	\includegraphics[width = 3.1in]{figure/chip_level_misaligned.png}
	\caption{Chip-level Misaligned}
	\label{Chip-level Misaligned}
\end{figure}
Define the result of the above situation to be $f^t(s_1, s_2, t_1, z)$
\begin{align}
	f^t(s_1, s_2, t_1, z) = (s_1 \cdot t_1)C_z^{N_c-1}(P^t_{k}, P^t_{k}, z) + (s_2 \cdot t_1) C_{N_c - z}^{N_c-1}(P^t_{k}, P^t_{k}, N_c - z)
\end{align}
Now, back to $g^t_{k_0}$
\begin{align}
	g^t_{k_0}
	&= \sum_{m = \floor{(-k_0)/N_c}, n = 0}^{\floor{m=(-k_0)/N_c} + N_s, n = N_s} f^t_{m\%N_s, m+1\%N_s, n, z\%N_c}
\end{align}

For $g^d_{k_0}$
\begin{align}
	\mathbb{E} \{g^d_{k_0}\} 
	&= \sum_{k' = 0}^{N_f-1} D^d_{\floor{(k'-z)/N_c}\%N_s} P^d_{(k'-z)\% N_c} \cdot D^t_{\floor{k'/N_c}} P^t_{k\%N_c}
\end{align}

We can also define
\begin{align}
	f^d_1(s_1, s_2, t_1, z) = (s_1 \cdot t_1)C_z^{N_c-1}(P^d_{k}, P^t_{k}, z) + (s_2 \cdot t_1) C_{N_c - z}^{N_c-1}(P^d_{k}, P^t_{k}, N_c - z)
\end{align}

Then $g_{k_0}^d$ would be 
\begin{align}
	g^d_{k_0}
	&= \sum_{m = \floor{(-k_0)/N_c}, n = 0}^{\floor{m=(-k_0)/N_c} + N_s, n = N_s} f^d(m, m+1, n, z\%N_c)
\end{align}
But for the expectation, we may need to change a little bit. Because we want to split the patten symbol rather than the received symbol.
\begin{align}
	g^d_{k_0}
	&= \sum_{m = \floor{(-k_0)/N_c}, n = 0}^{\floor{m=(-k_0)/N_c} + N_s, n = N_s} f^d(n, n+1, m, z\%N_c)
\end{align}
Then
\begin{align}
	\mathbb{E}\{g^d_{k_0}\} 
	&= \mathbb{E}\{\sum_{m = \floor{(-k_0)/N_c}, n = 0}^{\floor{m=(-k_0)/N_c} + N_s, n = N_s} f^d(n, n+1, m, z\%N_c)\}\\
	&= \sum_{m = \floor{(-k_0)/N_c}, n = 0}^{\floor{m=(-k_0)/N_c} + N_s, n = N_s} \mathbb{E}\{f^d(n, n+1, m, z\%N_c)\}\\
	&= 0
\end{align}
\begin{align}
	\mathbb{E} \{(g^d_{k_0})^2\}
	&= \bigg(C_{z\%N_c}^{N_c-1}(P^d_k, P^t_k, z\%N_c) + C_0^{z-1}(P^d_k, P^t_k, -z\%N_c)\bigg)^2 \cdot N_s
\end{align}

Because we can view it as a shifted training patten.
% subsubsection totally_misaligned (end)



% subsection description_of_output_of_matched_filter (end)
% section mathematical_model (end)

% \bibliographystyle{plain}
% \bibliography{bibliography.bib}




\end{document}